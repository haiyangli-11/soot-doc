\begin{slide}{Adding transformations to Soot (easy way)}
\vspace*{-0.1in}
\begin{enumerate}
\item Implement a \texttt{BodyTransformer} or\\ a \texttt{SceneTransformer}
\begin{itemize}
\item \texttt{internalTransform} method \\does the transformation
\end{itemize}
\item Choose a pack for your transformation (usually \texttt{jtp})
\item Write a \texttt{main} method that adds the transform to the pack, then
runs Soot's main
\item (Optional) If your transformation needs command-line options,
call \texttt{setDeclaredOptions()}
\end{enumerate}
\end{slide}

\begin{slide}{On {\tt Pack}s}
\vspace*{-0.2in}
Want to run a set of {\tt Transformer} objects with one method call.\\
$\qquad \Rightarrow$ Group them in a {\tt Pack}.

\vspace{0.1in}
Soot defines default {\tt Pack}s which are run automatically.
To add a {\tt Transformer} to the {\tt jtp} {\tt Pack}:
{\small
\begin{verbatim}
  Pack jtp = G.v().PackManager().
                     getPack("jtp");
  jtp.add(new Transform("jtp.nt", 
            new NullTransformer()));
  jtp.add(new Transform("jtp.nac", 
            new NullnessAnalysisColorer()));
\end{verbatim}}
\sablefootnote{soot.Pack}
\end{slide}

\begin{slide}{Extending Soot (hard way)}
Some don't like calling {\texttt{soot.Main.main()}}.\\
What does \texttt{main()} do?
\begin{enumerate}
\item \texttt{processCmdLine()}
\item \texttt{Scene.v().loadNecessaryClasses()}
\item \texttt{PackManager.v().runPacks()}
\item \texttt{PackManager.v().writeOutput()}
\end{enumerate}
You can do any or all of these yourself:
\begin{itemize}
\item \texttt{Options.v()} contains setter methods for all options
\end{itemize}
\end{slide}

\begin{slide}{Running Soot more than once}
\begin{itemize}
\item All Soot global variables are stored in \texttt{G.v()}
\item \texttt{G.reset()} re-initializes all of Soot
\end{itemize}
\end{slide}

\newcommand{\ph}[1]{\Rnode{#1}{\psframebox{\begin{tabular}{c}{#1}\end{tabular}}}}
\newcommand{\ar}[2]{\ncline{->}{#1}{#2}}
\newcommand{\arr}[2]{\ncline[linecolor=red]{->}{#1}{#2}}

\begin{slide}{Generating Jimple}
\psset{linearc=.25}
\centerline{
\begin{psmatrix}[colsep=.7,rowsep=.5]
\ovalnode{class}{\tt .class} & \ph{coffi} \\
&& \ph{jb} & \ovalnode{end}{Jimple}\\
\ovalnode{jimple}{\tt .jimple} & \Rnode{parser}{\psframebox{\begin{tabular}{c}Jimple\\parser\end{tabular}}} \\
\ar{class}{coffi}\ar{jimple}{parser}\nccurve[angleB=180]{->}{coffi}{jb}\nccurve[angleB=180]{->}{parser}{jb}\ar{jb}{end}
\end{psmatrix}
}
\end{slide}

\begin{slide}{Intra-procedural packs}
\vspace*{-5mm}
\psset{linearc=.25}
\hspace*{-3mm}
\begin{psmatrix}[colsep=.3,rowsep=.6]
&\ph{stp} & \ph{sop} & \ph{sap}&Shimple\\
\ovalnode{start}{Jimple}
&\ph{jtp} & \ph{jop} & \ph{jap}&Jimple\\
& \ph{bb} & \ph{bop} & \ph{tag}&Baf\\
& \ph{gb} & \ph{gop} &&Grimp\\
&&\ovalnode{Dava}{Dava}&\ovalnode{end}{Jasmin}&Output
\nccurve[angleA=90,angleB=180,linecolor=red]{->}{start}{stp}\ar{start}{jtp}\ar{jtp}{jop}\ar{jop}{jap}\ar{bb}{bop}\ar{bop}{tag}\ar{gb}{gop}\arr{gop}{Dava}\ar{stp}{sop}\ar{sop}{sap}
\ncloop[angleB=180,loopsize=-1,linecolor=red]{->}{sap}{jtp}
\ncloop[angleB=180,loopsize=-1,linecolor=red]{->}{jap}{bb}
\ncloop[angleB=180,loopsize=-1,linecolor=red]{->}{jap}{gb}
\arr{tag}{end}\arr{gop}{end}
\end{psmatrix}
\end{slide}


\begin{slide}{Soot Pack Naming Scheme}
\vspace*{-0.2in}
\[ w^? (j|s|b|g) (b|t|o|a)p \]
\sablefootnote{The p is sometimes silent.}

\begin{itemize}
\item w $\Rightarrow$ Whole-program phase
\item j, s, b, g $\Rightarrow$ Jimple, Shimple, Baf, Grimp
\item b, t, o, a $\Rightarrow$
\begin{itemize}
\item (b) Body creation
\item (t) User-defined transformations
\item (o) Optimizations with -O option
\item (a) Attribute generation
\end{itemize}
\end{itemize}
\end{slide}


\begin{slide}{Soot Packs (Jimple Body)}
\begin{description}
\item[jb] converts naive Jimple generated from bytecode into
typed Jimple with split variables
\end{description}
\end{slide}

\begin{slide}{Soot Packs (Jimple)}
\begin{description}
\item[jtp] performs user-defined intra-procedural transformations
\item[jop] performs intra-procedural optimizations
\begin{itemize}
\item CSE, PRE, constant propagation, \ldots
\end{itemize}
\item[jap] generates annotations using whole-program analyses
\begin{itemize}
\item null-pointer check
\item array bounds check
\item side-effect analysis
\end{itemize}
\end{description}
\end{slide}

\begin{slide}{Soot Packs (Back-end)}
\begin{description}
\item[bb] performs transformations to create Baf
\item[bop] performs user-defined Baf optimizations
\end{description}
\begin{description}
\item[gb] performs transformations to create Grimp
\item[gop] performs user-defined Grimp optimizations
\end{description}
\begin{description}
\item[tag] aggregates annotations into \\bytecode attributes
\end{description}
\end{slide}



